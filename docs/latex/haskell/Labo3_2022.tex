\haddockmoduleheading{Labo3_2022}
\label{module:Labo3_2022}
\haddockbeginheader
{\haddockverb\begin{verbatim}
module Labo3_2022 (
    last', reverse', any', min', map', takeWhile', listaNegPos, listaRepetidos,
    listaParejas, permuta, intercalado, cuestaPos, actualizarCabecera,
    meterCuesta, esCuesta
  ) where\end{verbatim}}
\haddockendheader

\section{EJERCICIO 1}
\subsection{A)}
\begin{haddockdesc}
\item[\begin{tabular}{@{}l}
last' :: {\char 91}a{\char 93} -> a
\end{tabular}]
{\haddockbegindoc
\section*{Descripción}
Coge el último elemento de una lista dada\par
\subsection*{Ejemplos}
\begin{quote}
{\haddockverb\begin{verbatim}
>>> last' []
Lista vacia

\end{verbatim}}
\end{quote}
\begin{quote}
{\haddockverb\begin{verbatim}
>>> last [1,2,3,4]
4

\end{verbatim}}
\end{quote}}
\end{haddockdesc}
\subsection{B)}
\begin{haddockdesc}
\item[\begin{tabular}{@{}l}
reverse' :: {\char 91}a{\char 93} -> {\char 91}a{\char 93}
\end{tabular}]
{\haddockbegindoc
\section*{Descripción}
Hace la lista inversa de una dada\par
\subsection*{Ejemplos}
\begin{quote}
{\haddockverb\begin{verbatim}
>>> reverse' [1,2,3,4]
[4,3,2,1]

\end{verbatim}}
\end{quote}}
\end{haddockdesc}
\subsection{C)}
\begin{haddockdesc}
\item[\begin{tabular}{@{}l}
any' :: (a -> Bool) -> {\char 91}a{\char 93} -> Bool
\end{tabular}]
{\haddockbegindoc
\section*{Descripción}
Devuelve true si todos los elementos de la lista cumplen una condición\par
\subsection*{Ejemplos}
\begin{quote}
{\haddockverb\begin{verbatim}
>>> any' (<10) [1,2,3,4]
True

\end{verbatim}}
\end{quote}
\begin{quote}
{\haddockverb\begin{verbatim}
>>> any' (<10) [10, 20, 30, 5]
True

\end{verbatim}}
\end{quote}
\begin{quote}
{\haddockverb\begin{verbatim}
>>> any' (>10) [1,2, 3, 4]
False

\end{verbatim}}
\end{quote}}
\end{haddockdesc}
\subsection{D)}
\begin{haddockdesc}
\item[\begin{tabular}{@{}l}
min' :: Ord a => {\char 91}a{\char 93} -> a
\end{tabular}]
{\haddockbegindoc
\section*{Descripción}
Devuelve el menor elemento de una lista dada\par
\subsection*{Ejemplos}
\begin{quote}
{\haddockverb\begin{verbatim}
>>> min' []
Lista vacia

\end{verbatim}}
\end{quote}
\begin{quote}
{\haddockverb\begin{verbatim}
>>> min' [2,5,8,2,0]
0

\end{verbatim}}
\end{quote}}
\end{haddockdesc}
\subsection{E)}
\begin{haddockdesc}
\item[\begin{tabular}{@{}l}
map' :: (a -> b) -> {\char 91}a{\char 93} -> {\char 91}b{\char 93}
\end{tabular}]
{\haddockbegindoc
\section*{Descripción}
Genera una lista de elementos resultantes de aplicar un filtro a otra lista de elementos\par
\subsection*{Ejemplo}
\begin{quote}
{\haddockverb\begin{verbatim}
>>> map' (*3) [1,2,3,4]
[3,6,9,12]

\end{verbatim}}
\end{quote}}
\end{haddockdesc}
\subsection{F)}
\begin{haddockdesc}
\item[\begin{tabular}{@{}l}
takeWhile' :: (a -> Bool) -> {\char 91}a{\char 93} -> {\char 91}a{\char 93}
\end{tabular}]
{\haddockbegindoc
\section*{Descripción}
Coge elementos de la lista dada hasta que la condición falla\par
\subsection*{Ejemplos}
\begin{quote}
{\haddockverb\begin{verbatim}
>>> takeWhile' (<3) [1,2,3,4,5]
[1,2]

\end{verbatim}}
\end{quote}
\begin{quote}
{\haddockverb\begin{verbatim}
>>> takeWhile' (>3) [1,2,3,4,5]
[]

\end{verbatim}}
\end{quote}}
\end{haddockdesc}
\section{EJERCICIO 2}
\begin{haddockdesc}
\item[\begin{tabular}{@{}l}
listaNegPos :: {\char 91}Integer{\char 93}
\end{tabular}]
{\haddockbegindoc
\section*{Descripción}
Genera una lista alternada de elementos del 1 al 100 de forma {\char 91}1, -2,..{\char 93}\par
\begin{quote}
{\haddockverb\begin{verbatim}
>>> listaNegPos
[1,-2,3,-4,5,-6,7,-8,9,-10,11,-12,13,-14,15,-16,17,-18,19,-20,21,-22,23,-24,25,-26,27,-28,29,-30,31,-32,33,-34,35,-36,37,-38,39,-40,41,-42,43,-44,45,-46,47,-48,49,-50,51,-52,53,-54,55,-56,57,-58,59,-60,61,-62,63,-64,65,-66,67,-68,69,-70,71,-72,73,-74,75,-76,77,-78,79,-80,81,-82,83,-84,85,-86,87,-88,89,-90,91,-92,93,-94,95,-96,97,-98,99,-100]

\end{verbatim}}
\end{quote}}
\end{haddockdesc}
\section{EJERCICIO 3}
\begin{haddockdesc}
\item[\begin{tabular}{@{}l}
listaRepetidos :: {\char 91}{\char 91}Int{\char 93}{\char 93}
\end{tabular}]
{\haddockbegindoc
\section*{Descripción}
Genera una lista de listas cuyo contenido sera el numero x repetido x veces\par
\begin{quote}
{\haddockverb\begin{verbatim}
>>> take 5 (listaRepetidos)
[[],[1],[2,2],[3,3,3],[4,4,4,4]]

\end{verbatim}}
\end{quote}}
\end{haddockdesc}
\section{EJERCICIO 4}
\begin{haddockdesc}
\item[\begin{tabular}{@{}l}
listaParejas :: Integral a => {\char 91}(a, a){\char 93}
\end{tabular}]
{\haddockbegindoc
\section*{Descripción}
Genera parejas ordenadas que cumplen x + y = z, en orden de z\par
\begin{quote}
{\haddockverb\begin{verbatim}
>>> take 10 (listaParejas)
[(0,0),(0,1),(1,0),(0,2),(1,1),(2,0),(0,3),(1,2),(2,1),(3,0)]

\end{verbatim}}
\end{quote}}
\end{haddockdesc}
\section{EJERCICIO 5}
\begin{haddockdesc}
\item[\begin{tabular}{@{}l}
permuta :: {\char 91}a{\char 93} -> {\char 91}{\char 91}a{\char 93}{\char 93}
\end{tabular}]
{\haddockbegindoc
\section*{Descripción}
Genera todas las permutaciones posibles de una lista dada\par
\subsection*{Ejemplo}
\begin{quote}
{\haddockverb\begin{verbatim}
>>> permuta [1,2,3]
[[1,2,3],[2,1,3],[2,3,1],[1,3,2],[3,1,2],[3,2,1]]

\end{verbatim}}
\end{quote}}
\end{haddockdesc}
\begin{haddockdesc}
\item[\begin{tabular}{@{}l}
intercalado :: a -> {\char 91}a{\char 93} -> {\char 91}{\char 91}a{\char 93}{\char 93}
\end{tabular}]
{\haddockbegindoc
\section*{Descripción}
Devuelve una lista de listas con el elemento dado en cada posición posible de la lista dada\par
\subsection*{Ejemplo}
\begin{quote}
{\haddockverb\begin{verbatim}
>>> intercalado 3 [1,2]
[[3,1,2],[1,3,2],[1,2,3]]

\end{verbatim}}
\end{quote}}
\end{haddockdesc}
\section{EJERCICIO 6}
\begin{haddockdesc}
\item[\begin{tabular}{@{}l}
cuestaPos :: Ord a => {\char 91}a{\char 93} -> {\char 91}(a, Int){\char 93}
\end{tabular}]
{\haddockbegindoc
\section*{Descripción}
Devuelve una lista con los elementos considerados cuestas y su posicion\par
\subsection*{Ejemplo}
\begin{quote}
{\haddockverb\begin{verbatim}
>>> cuestaPos [-3,-2,1,0,-1,1]
[(-2,1),(1,2),(1,5)]

\end{verbatim}}
\end{quote}}
\end{haddockdesc}
\begin{haddockdesc}
\item[\begin{tabular}{@{}l}
actualizarCabecera :: {\char 91}(a, Int){\char 93} -> a -> {\char 91}(a, Int){\char 93}
\end{tabular}]
{\haddockbegindoc
\section*{Descripción}
Devuelve una lista con la cabecera actualizada\par
\subsection*{Ejemplo}
\begin{quote}
{\haddockverb\begin{verbatim}
>>> actualizarCabecera [(3,0)] 2
[(2,1)]

\end{verbatim}}
\end{quote}}
\end{haddockdesc}
\begin{haddockdesc}
\item[\begin{tabular}{@{}l}
meterCuesta :: {\char 91}(a, Int){\char 93} -> a -> {\char 91}(a, Int){\char 93}
\end{tabular}]
{\haddockbegindoc
\section*{Descripción}
Añade una cuesta al listado y actualiza la cabecera\par
\subsection*{Ejemplo}
\begin{quote}
{\haddockverb\begin{verbatim}
>>> meterCuesta [(3,0)] 2
[(2,1),(2,1)]

\end{verbatim}}
\end{quote}}
\end{haddockdesc}
\begin{haddockdesc}
\item[\begin{tabular}{@{}l}
esCuesta :: Ord a => a -> a -> Bool
\end{tabular}]
{\haddockbegindoc
\section*{Descripción}
comprueba si un elemento es cuesta\par
\subsection*{Ejemplo}
\begin{quote}
{\haddockverb\begin{verbatim}
>>> esCuesta 1 0
False

\end{verbatim}}
\end{quote}
\begin{quote}
{\haddockverb\begin{verbatim}
>>> esCuesta 2 5
True

\end{verbatim}}
\end{quote}}
\end{haddockdesc}