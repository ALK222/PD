\haddockmoduleheading{Labo2_2022}
\label{module:Labo2_2022}
\haddockbeginheader
{\haddockverb\begin{verbatim}
module Labo2_2022 (
    listaPares, listaParesCuadrados, listaPotencias, sumaMenores,
    primerPrimoMayor, isPrime, iguales, menorA, menor, mayorA, pt,
    filter2, partition, mapx, filter1, filters
  ) where\end{verbatim}}
\haddockendheader

\section{EJERCICIO 1}
\subsection{A)}
\begin{haddockdesc}
\item[\begin{tabular}{@{}l}
listaPares :: Integral a => a -> {\char 91}a{\char 93}
\end{tabular}]
{\haddockbegindoc
\section*{Descripción}
Genera una lista de números pares hasta el número dado.\par
\subsection*{Ejemplo:}
\begin{quote}
{\haddockverb\begin{verbatim}
>>> listaPares 9
[0,2,4,6,8]

\end{verbatim}}
\end{quote}}
\end{haddockdesc}
\subsection{B)}
\begin{haddockdesc}
\item[\begin{tabular}{@{}l}
listaParesCuadrados :: Integral a => a -> {\char 91}(a, a){\char 93}
\end{tabular}]
{\haddockbegindoc
\section*{Descripción}
Coje un números y devuelve una lista con los cuadrados desde 0 hasta el número en orden inverso y emparejado con su número inicial\par
\subsection*{Ejemplos}
\begin{quote}
{\haddockverb\begin{verbatim}
>>> listaParesCuadrados 3
[(3,9),(2,4),(1,1),(0,0)]

\end{verbatim}}
\end{quote}}
\end{haddockdesc}
\subsection{C)}
\begin{haddockdesc}
\item[\begin{tabular}{@{}l}
listaPotencias :: Integral a => Int -> {\char 91}a{\char 93}
\end{tabular}]
{\haddockbegindoc
\section*{Descripción}
Genera una lista con las potencias de 3 desde 0 hasta el numero dado\par
\subsection*{Ejemplo:}
\begin{quote}
{\haddockverb\begin{verbatim}
>>> listaPotencias 3
[1,3,9]

\end{verbatim}}
\end{quote}}
\end{haddockdesc}
\subsection{D)}
\begin{haddockdesc}
\item[\begin{tabular}{@{}l}
sumaMenores :: Int -> Int
\end{tabular}]
{\haddockbegindoc
\section*{Descripción}
Sumatorio de los números menores que el dado que sean multiplos de 5 o 3\par
\subsection*{Ejemplos}
\begin{quote}
{\haddockverb\begin{verbatim}
>>> sumaMenores 10
33

\end{verbatim}}
\end{quote}}
\end{haddockdesc}
\subsection{E)}
\begin{haddockdesc}
\item[\begin{tabular}{@{}l}
primerPrimoMayor :: Integral a => a -> a
\end{tabular}]
{\haddockbegindoc
\section*{Descripción}
calcula el siguiente número primo a uno dado.\par
Usa \haddockid{isPrime}\par
\subsection*{Ejemplos}
\begin{quote}
{\haddockverb\begin{verbatim}
>>> primerPrimoMayor 10
11

\end{verbatim}}
\end{quote}}
\end{haddockdesc}
\begin{haddockdesc}
\item[\begin{tabular}{@{}l}
isPrime :: Integral a => a -> Bool
\end{tabular}]
{\haddockbegindoc
\section*{Descripción}
Comprueba si un número es primo\par
\subsection*{Ejemplos}
\begin{quote}
{\haddockverb\begin{verbatim}
>>> isPrime 10
False

\end{verbatim}}
\end{quote}
\begin{quote}
{\haddockverb\begin{verbatim}
>>> isPrime 5
True

\end{verbatim}}
\end{quote}}
\end{haddockdesc}
\section{EJERCICIO 2}
\subsection{A)}
\begin{haddockdesc}
\item[\begin{tabular}{@{}l}
iguales :: Eq b => (a -> b) -> (a -> b) -> a -> a -> Bool
\end{tabular}]
{\haddockbegindoc
\section*{Descripción}
Comprueba si dos funciones son iguales para un ranfo dado de valores\par
\subsection*{Ejemplos}
\begin{quote}
{\haddockverb\begin{verbatim}
>>> iguales primerPrimoMayor sumaMenores 0 10
False

\end{verbatim}}
\end{quote}
\begin{quote}
{\haddockverb\begin{verbatim}
>>> iguales sumaMenores sumaMenores 0 10
True

\end{verbatim}}
\end{quote}}
\end{haddockdesc}
\subsection{B)}
\begin{haddockdesc}
\item[\begin{tabular}{@{}l}
menorA :: Integral a => a -> a -> (a -> Bool) -> a
\end{tabular}]
{\haddockbegindoc
\section*{Descripción}
Devuelve el menor numero mayor que n que compla una función dada\par
\subsection*{Ejemplos}
\begin{quote}
{\haddockverb\begin{verbatim}
>>> menorA 22 128 isPrime
23

\end{verbatim}}
\end{quote}
\begin{quote}
{\haddockverb\begin{verbatim}
>>> menorA 14 16 isPrime
Prelude.head: empty list

\end{verbatim}}
\end{quote}}
\end{haddockdesc}
\subsection{C)}
\begin{haddockdesc}
\item[\begin{tabular}{@{}l}
menor :: (Num p, Enum p) => p -> (p -> Bool) -> p
\end{tabular}]
{\haddockbegindoc
\section*{Descripción}
Devuelve el menor numero mayor que n que compla una función dada\par
\subsection*{Ejemplos}
\begin{quote}
{\haddockverb\begin{verbatim}
>>> menor 2077 isPrime
2081

\end{verbatim}}
\end{quote}}
\end{haddockdesc}
\subsection{D)}
\begin{haddockdesc}
\item[\begin{tabular}{@{}l}
mayorA :: Enum p => p -> p -> (p -> Bool) -> p
\end{tabular}]
{\haddockbegindoc
\section*{Descripción}
Mayor número en un intervalo dado que cumple una función dada\par
\subsection*{Ejemplos}
\begin{quote}
{\haddockverb\begin{verbatim}
>>> mayorA 10 20 isPrime
19

\end{verbatim}}
\end{quote}
\begin{quote}
{\haddockverb\begin{verbatim}
>>> mayorA 14 16 isPrime
Prelude.last: empty list

\end{verbatim}}
\end{quote}}
\end{haddockdesc}
\subsection{E)}
\begin{haddockdesc}
\item[\begin{tabular}{@{}l}
pt :: Enum p => p -> p -> (p -> Bool) -> Bool
\end{tabular}]
{\haddockbegindoc
\section*{Descripción}
Comprueba si todos los elementos de un intervalo verifican una función\par
\subsection*{Ejemplos}
\begin{quote}
{\haddockverb\begin{verbatim}
>>> pt 14 16 isPrime
False

\end{verbatim}}
\end{quote}
\begin{quote}
{\haddockverb\begin{verbatim}
>>> pt 2 3 isPrime
True

\end{verbatim}}
\end{quote}}
\end{haddockdesc}
\section{EJERCICIO 3}
\subsection{A)}
\begin{haddockdesc}
\item[\begin{tabular}{@{}l}
filter2 :: {\char 91}a{\char 93} -> (a -> b) -> (a -> b) -> {\char 91}{\char 91}b{\char 93}{\char 93}
\end{tabular}]
{\haddockbegindoc
\section*{Descripción}
Devuelve dos listas con los elementos de una lista dada que cumplen dos funciones\par
\subsection*{Ejemplos}
\begin{quote}
{\haddockverb\begin{verbatim}
>>> filter2 [1..10] odd even
([1,3,5,7,9],[2,4,6,8,10])

\end{verbatim}}
\end{quote}}
\end{haddockdesc}
\subsection{B)}
\begin{haddockdesc}
\item[\begin{tabular}{@{}l}
partition :: (a -> Bool) -> {\char 91}a{\char 93} -> ({\char 91}a{\char 93}, {\char 91}a{\char 93})
\end{tabular}]
{\haddockbegindoc
\section*{Descripción}
Parte una lista dada entre elementos que verifican una función y otros que no\par
\subsection*{Ejemplos}
\begin{quote}
{\haddockverb\begin{verbatim}
>>> partition isPrime [0..20]
([2,3,5,7,11,13,17,19],[0,1,4,6,8,9,10,12,14,15,16,18,20])

\end{verbatim}}
\end{quote}}
\end{haddockdesc}
\subsection{C)}
\begin{haddockdesc}
\item[\begin{tabular}{@{}l}
mapx :: t -> {\char 91}t -> b{\char 93} -> {\char 91}b{\char 93}
\end{tabular}]
{\haddockbegindoc
\section*{Descripción}
Devuelve una lista de valores resultado de aplicar a un valor un listado de funciones. Todos los tipos de las funciones de la lista deben ser iguales\par
\subsection*{Ejemplos}
\begin{quote}
{\haddockverb\begin{verbatim}
>>> mapx 10 [even, odd]
[True,False]

\end{verbatim}}
\end{quote}
\begin{quote}
{\haddockverb\begin{verbatim}
>>> mapx 10 [primerPrimoMayor, sumaMenores]
[11,33]

\end{verbatim}}
\end{quote}}
\end{haddockdesc}
\subsection{D)}
\begin{haddockdesc}
\item[\begin{tabular}{@{}l}
filter1 :: {\char 91}{\char 91}a{\char 93}{\char 93} -> (a -> Bool) -> {\char 91}{\char 91}a{\char 93}{\char 93}
\end{tabular}]
{\haddockbegindoc
\section*{Description}
Devuelve una lista con listas de valores que cumplen la propiedad dada\par
\subsection*{Ejemplo}
\begin{quote}
{\haddockverb\begin{verbatim}
>>> filter1 [[1], [1,2,3,4], [2077,2467, 2277]] isPrime
[[],[2,3],[2467]]

\end{verbatim}}
\end{quote}}
\end{haddockdesc}
\subsection{E)}
\begin{haddockdesc}
\item[\begin{tabular}{@{}l}
filters :: {\char 91}a{\char 93} -> {\char 91}a -> Bool{\char 93} -> {\char 91}{\char 91}a{\char 93}{\char 93}
\end{tabular}]
{\haddockbegindoc
\section*{Descripción}
Devuelve una lista con las listas de numeros que cumplen unas funciones dadas en formato de lista\par
\subsection*{Ejemplos}
\begin{quote}
{\haddockverb\begin{verbatim}
>>> filters [1..10] [odd, even, isPrime]
[[1,3,5,7,9],[2,4,6,8,10],[2,3,5,7]]

\end{verbatim}}
\end{quote}}
\end{haddockdesc}