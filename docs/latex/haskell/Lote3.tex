\haddockmoduleheading{Lote3}
\label{module:Lote3}
\haddockbeginheader
{\haddockverb\begin{verbatim}
module Lote3 (
    Complejo(C), Arbol(N), showTree, listaHojas, listaNodos, repMax,
    Medible(medida), Exp(Div, Por, Menos, Mas, Num), eval, cuentaOps, cambiaOps,
    operandos, Punto(P), pointSum, Direccion(DERECHA, IZQUIERDA, ABAJO, ARRIBA),
    mueve, destino, trayectoria, inferior, Conjunto(Con), toC, interseccion,
    union, mapSet, Asec(Acons, Void), long, nElem, separa, showSec, palabras,
    palabras', getInt, promedia, promediaAux
  ) where\end{verbatim}}
\haddockendheader

\section{EJERCICIO 1}
\begin{haddockdesc}
\item[\begin{tabular}{@{}l}
data Complejo
\end{tabular}]
{\haddockbegindoc
Representación de un numero complejo\par
\enspace \emph{Constructors}\par
\haddockbeginconstrs
\haddockdecltt{=} & \haddockdecltt{C Float Float} & \\
\end{tabulary}\par}
\end{haddockdesc}
\begin{haddockdesc}
\item[\begin{tabular}{@{}l}
instance Num Complejo
\end{tabular}]
{\haddockbegindoc
Implementación especial para \haddockid{(+)} \haddockid{(-)} y \haddockid{(*)}\par}
\end{haddockdesc}
\begin{haddockdesc}
\item[\begin{tabular}{@{}l}
instance Show Complejo
\end{tabular}]
{\haddockbegindoc
Representa el número como "a (+-) bi"\par}
\end{haddockdesc}
\begin{haddockdesc}
\item[\begin{tabular}{@{}l}
instance Eq Complejo
\end{tabular}]
{\haddockbegindoc
Derivación estandar de Eq\par}
\end{haddockdesc}
\section{EJERCICIO 2}
\begin{haddockdesc}
\item[\begin{tabular}{@{}l}
data Arbol a
\end{tabular}]
{\haddockbegindoc
\enspace \emph{Constructors}\par
\haddockbeginconstrs
\haddockdecltt{=} & \haddockdecltt{N a {\char 91}Arbol a{\char 93}} & \\
\end{tabulary}\par}
\end{haddockdesc}
\begin{haddockdesc}
\item[\begin{tabular}{@{}l}
instance Show a => Show Arbol a\\instance Eq a => Eq Arbol a\\instance Ord a => Ord Arbol a
\end{tabular}]
\end{haddockdesc}
\begin{haddockdesc}
\item[\begin{tabular}{@{}l}
showTree :: Show a => Int -> Int -> Arbol a -> {\char 91}Char{\char 93}
\end{tabular}]
\item[\begin{tabular}{@{}l}
listaHojas :: Arbol a -> {\char 91}a{\char 93}
\end{tabular}]
\item[\begin{tabular}{@{}l}
listaNodos :: Arbol a -> {\char 91}a{\char 93}
\end{tabular}]
\item[\begin{tabular}{@{}l}
repMax :: Ord a => Arbol a -> Arbol a
\end{tabular}]
\end{haddockdesc}
\section{EJERCICIO 3}
\begin{haddockdesc}
\item[\begin{tabular}{@{}l}
class Medible a where
\end{tabular}]
{\haddockbegindoc
Clase para medir otros tipos de datos\par
\haddockpremethods{}\emph{Methods}
\begin{haddockdesc}
\item[\begin{tabular}{@{}l}
medida
\end{tabular}]
{\haddockbegindoc
\haddockbeginargs
\haddockdecltt{::} & \haddockdecltt{a} & \\
\haddockdecltt{->} & \haddockdecltt{Int} & Coge el dato dado y saca una medida de él \\
\end{tabulary}\par}
\end{haddockdesc}}
\end{haddockdesc}
\begin{haddockdesc}
\item[\begin{tabular}{@{}l}
instance Medible Bool
\end{tabular}]
{\haddockbegindoc
Mediremos True como 0  y false como 1\par}
\end{haddockdesc}
\begin{haddockdesc}
\item[\begin{tabular}{@{}l}
instance Medible a => Medible {\char 91}a{\char 93}
\end{tabular}]
{\haddockbegindoc
Suma todas las medidas de la lista\par}
\end{haddockdesc}
\begin{haddockdesc}
\item[\begin{tabular}{@{}l}
instance (Medible a, Medible b) => Medible (a, b)
\end{tabular}]
{\haddockbegindoc
Suma la medida de los dos elementos dados\par}
\end{haddockdesc}
\section{EJERCICIO 4}
\begin{haddockdesc}
\item[\begin{tabular}{@{}l}
data Exp
\end{tabular}]
{\haddockbegindoc
\enspace \emph{Constructors}\par
\haddockbeginconstrs
\haddockdecltt{=} & \haddockdecltt{Num Int} & \\
\haddockdecltt{|} & \haddockdecltt{Mas Exp Exp} & \\
\haddockdecltt{|} & \haddockdecltt{Menos Exp Exp} & \\
\haddockdecltt{|} & \haddockdecltt{Por Exp Exp} & \\
\haddockdecltt{|} & \haddockdecltt{Div Exp Exp} & \\
\end{tabulary}\par}
\end{haddockdesc}
\begin{haddockdesc}
\item[\begin{tabular}{@{}l}
instance Show Exp
\end{tabular}]
\end{haddockdesc}
\begin{haddockdesc}
\item[\begin{tabular}{@{}l}
eval :: Exp -> Int
\end{tabular}]
\item[\begin{tabular}{@{}l}
cuentaOps :: Exp -> Int
\end{tabular}]
\item[\begin{tabular}{@{}l}
cambiaOps :: Exp -> Exp
\end{tabular}]
\item[\begin{tabular}{@{}l}
operandos :: Exp -> {\char 91}Int{\char 93}
\end{tabular}]
\end{haddockdesc}
\section{EJERCICIO 5}
\begin{haddockdesc}
\item[\begin{tabular}{@{}l}
data Punto
\end{tabular}]
{\haddockbegindoc
Representación de un punto de forma (x,y)\par
\enspace \emph{Constructors}\par
\haddockbeginconstrs
\haddockdecltt{=} & \haddockdecltt{P} & Constructor del punto \\
                                        & \qquad \haddockdecltt{Int} & Coordenada X \\
                                        & \qquad \haddockdecltt{Int} & Coordenada Y \\
\end{tabulary}\par}
\end{haddockdesc}
\begin{haddockdesc}
\item[\begin{tabular}{@{}l}
instance Show Punto
\end{tabular}]
{\haddockbegindoc
Muestra el punto como "(a,b)""\par}
\end{haddockdesc}
\begin{haddockdesc}
\item[\begin{tabular}{@{}l}
pointSum :: (Ord a, Num a) => a -> a -> a
\end{tabular}]
{\haddockbegindoc
\section*{Descripción}
Suma un numero a una parte de una coordenada dada. Tiene como rango {\char 91}0 .. 100{\char 93}\par
\subsection*{Ejemplos}
\begin{quote}
{\haddockverb\begin{verbatim}
>>> pointSum 100 10
9

\end{verbatim}}
\end{quote}
\begin{quote}
{\haddockverb\begin{verbatim}
>>> pointSum 0 (-1)
100

\end{verbatim}}
\end{quote}
\begin{quote}
{\haddockverb\begin{verbatim}
>>> pointSum 100 1
0

\end{verbatim}}
\end{quote}}
\end{haddockdesc}
\begin{haddockdesc}
\item[\begin{tabular}{@{}l}
data Direccion
\end{tabular}]
{\haddockbegindoc
Representación de una dirección\par
\enspace \emph{Constructors}\par
\haddockbeginconstrs
\haddockdecltt{=} & \haddockdecltt{ARRIBA} & \\
\haddockdecltt{|} & \haddockdecltt{ABAJO} & \\
\haddockdecltt{|} & \haddockdecltt{IZQUIERDA} & \\
\haddockdecltt{|} & \haddockdecltt{DERECHA} & \\
\end{tabulary}\par}
\end{haddockdesc}
\begin{haddockdesc}
\item[\begin{tabular}{@{}l}
instance Show Direccion
\end{tabular}]
{\haddockbegindoc
Derivación estandar de Show\par}
\end{haddockdesc}
\begin{haddockdesc}
\item[\begin{tabular}{@{}l}
instance Eq Direccion
\end{tabular}]
{\haddockbegindoc
Derivación estandar de Eq\par}
\end{haddockdesc}
\begin{haddockdesc}
\item[\begin{tabular}{@{}l}
instance Ord Direccion
\end{tabular}]
{\haddockbegindoc
Derivación estandar de Ord\par}
\end{haddockdesc}
\subsection{A)}
\begin{haddockdesc}
\item[\begin{tabular}{@{}l}
mueve :: Punto -> Direccion -> Punto
\end{tabular}]
{\haddockbegindoc
\section*{Descripción}
Mueve un punto en la dirección dada\par
\subsection*{Ejemplos}
\begin{quote}
{\haddockverb\begin{verbatim}
>>> mueve (P 0 1) IZQUIERDA
(100,1)

\end{verbatim}}
\end{quote}
\begin{quote}
{\haddockverb\begin{verbatim}
>>> mueve (P 0 1) DERECHA
(1,1)

\end{verbatim}}
\end{quote}
\begin{quote}
{\haddockverb\begin{verbatim}
>>> mueve (P 0 1) ARRIBA
(0,2)

\end{verbatim}}
\end{quote}
\begin{quote}
{\haddockverb\begin{verbatim}
>>> mueve (P 0 1) ABAJO
(0,0)

\end{verbatim}}
\end{quote}}
\end{haddockdesc}
\subsection{B)}
\begin{haddockdesc}
\item[\begin{tabular}{@{}l}
destino :: Punto -> {\char 91}Direccion{\char 93} -> Punto
\end{tabular}]
{\haddockbegindoc
\section*{Descripción}
Mueve un punto en las direcciones dadas\par
\subsection*{Ejemplos}
\begin{quote}
{\haddockverb\begin{verbatim}
>>> destino (P 0 0) [ARRIBA, ARRIBA, ABAJO, ABAJO, IZQUIERDA, DERECHA, IZQUIERDA, DERECHA]
(0,0)

\end{verbatim}}
\end{quote}
\begin{quote}
{\haddockverb\begin{verbatim}
>>> destino (P 0 0) [ARRIBA, ARRIBA, ABAJO, IZQUIERDA, IZQUIERDA, DERECHA]
(100,1)

\end{verbatim}}
\end{quote}}
\end{haddockdesc}
\subsection{C)}
\begin{haddockdesc}
\item[\begin{tabular}{@{}l}
trayectoria :: Punto -> {\char 91}Direccion{\char 93} -> {\char 91}Punto{\char 93}
\end{tabular}]
{\haddockbegindoc
\section*{Descripción}
Mueve un punto en las direcciones dadas mostrando los puntos por los que se han pasado\par
\subsection*{Ejemplos}
\begin{quote}
{\haddockverb\begin{verbatim}
>>> trayectoria (P 0 0) [ARRIBA, ARRIBA, ABAJO, ABAJO, IZQUIERDA, DERECHA, IZQUIERDA, DERECHA]
[(0,0),(0,1),(0,2),(0,1),(0,0),(100,0),(0,0),(100,0),(0,0)]

\end{verbatim}}
\end{quote}
\begin{quote}
{\haddockverb\begin{verbatim}
>>> trayectoria (P 0 0) [ARRIBA, ARRIBA, ABAJO, IZQUIERDA, IZQUIERDA, DERECHA]
[(0,0),(0,1),(0,2),(0,1),(100,1),(99,1),(100,1)]

\end{verbatim}}
\end{quote}}
\end{haddockdesc}
\begin{haddockdesc}
\item[\begin{tabular}{@{}l}
inferior :: {\char 91}Direccion{\char 93} -> {\char 91}Direccion{\char 93} -> Bool
\end{tabular}]
\end{haddockdesc}
\section{EJERCICIO 6}
\begin{haddockdesc}
\item[\begin{tabular}{@{}l}
data Conjunto a
\end{tabular}]
{\haddockbegindoc
\enspace \emph{Constructors}\par
\haddockbeginconstrs
\haddockdecltt{=} & \haddockdecltt{Con Int {\char 91}a{\char 93}} & \\
\end{tabulary}\par}
\end{haddockdesc}
\begin{haddockdesc}
\item[\begin{tabular}{@{}l}
instance Eq a => Eq Conjunto a\\instance Ord a => Ord Conjunto a
\end{tabular}]
\end{haddockdesc}
\begin{haddockdesc}
\item[\begin{tabular}{@{}l}
toC :: Eq a => {\char 91}a{\char 93} -> Conjunto a
\end{tabular}]
\item[\begin{tabular}{@{}l}
interseccion :: Eq a => Conjunto a -> Conjunto a -> Conjunto a
\end{tabular}]
\item[\begin{tabular}{@{}l}
union :: Eq a => Conjunto a -> Conjunto a -> Conjunto a
\end{tabular}]
\item[\begin{tabular}{@{}l}
mapSet :: (Eq a, Eq b) => (a -> b) -> Conjunto a -> Conjunto b
\end{tabular}]
\end{haddockdesc}
\section{EJERCICIO 7}
\begin{haddockdesc}
\item[\begin{tabular}{@{}l}
data Asec a b
\end{tabular}]
{\haddockbegindoc
\enspace \emph{Constructors}\par
\haddockbeginconstrs
\haddockdecltt{=} & \haddockdecltt{Void} & \\
\haddockdecltt{|} & \haddockdecltt{Acons a Asec b a} & \\
\end{tabulary}\par}
\end{haddockdesc}
\begin{haddockdesc}
\item[\begin{tabular}{@{}l}
instance (Show a, Show b) => Show Asec a b\\instance (Eq a, Eq b) => Eq Asec a b\\instance (Ord a, Ord b) => Ord Asec a b
\end{tabular}]
\end{haddockdesc}
\begin{haddockdesc}
\item[\begin{tabular}{@{}l}
long :: Num p => Asec a b -> p
\end{tabular}]
\item[\begin{tabular}{@{}l}
nElem :: Int -> Asec a b -> Either a b
\end{tabular}]
\item[\begin{tabular}{@{}l}
separa :: Asec a b -> ({\char 91}a{\char 93}, {\char 91}b{\char 93})
\end{tabular}]
\item[\begin{tabular}{@{}l}
showSec :: (Show a, Show b) => Asec a b -> String
\end{tabular}]
\end{haddockdesc}
\section{EJERCICIO 8}
\begin{haddockdesc}
\item[\begin{tabular}{@{}l}
palabras :: String -> IO Int
\end{tabular}]
\item[\begin{tabular}{@{}l}
palabras' :: IO ()
\end{tabular}]
\item[\begin{tabular}{@{}l}
getInt :: IO Int
\end{tabular}]
\item[\begin{tabular}{@{}l}
promedia :: IO ()
\end{tabular}]
\item[\begin{tabular}{@{}l}
promediaAux :: {\char 91}Int{\char 93} -> IO ()
\end{tabular}]
\end{haddockdesc}