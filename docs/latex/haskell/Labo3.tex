\haddockmoduleheading{Labo3}
\label{module:Labo3}
\haddockbeginheader
{\haddockverb\begin{verbatim}
module Labo3 (
    last', reverse', all', min', map', filter', takeWhile',
    listaNegPos, listaParejas', listaParejas, sufijos, sublistas,
    permuta, intercalado, sumandos, sumandos'
  ) where\end{verbatim}}
\haddockendheader

\section{EJERCICIO 1}
\subsection{A)}
\begin{haddockdesc}
\item[\begin{tabular}{@{}l}
last' :: {\char 91}a{\char 93} -> a
\end{tabular}]
{\haddockbegindoc
\section*{Descripción}
Coge el último elemento de una lista dada\par
\subsection*{Ejemplos}
\begin{quote}
{\haddockverb\begin{verbatim}
>>> last' []
Lista vacia

\end{verbatim}}
\end{quote}
\begin{quote}
{\haddockverb\begin{verbatim}
>>> last [1,2,3,4]
4

\end{verbatim}}
\end{quote}}
\end{haddockdesc}
\subsection{B)}
\begin{haddockdesc}
\item[\begin{tabular}{@{}l}
reverse' :: {\char 91}a{\char 93} -> {\char 91}a{\char 93}
\end{tabular}]
{\haddockbegindoc
\section*{Descripción}
Hace la lista inversa de una dada\par
\subsection*{Ejemplos}
\begin{quote}
{\haddockverb\begin{verbatim}
>>> reverse' [1,2,3,4]
[4,3,2,1]

\end{verbatim}}
\end{quote}}
\end{haddockdesc}
\subsection{C)}
\begin{haddockdesc}
\item[\begin{tabular}{@{}l}
all' :: (a -> Bool) -> {\char 91}a{\char 93} -> Bool
\end{tabular}]
{\haddockbegindoc
\section*{Descripción}
Devuelve true si todos los elementos de la lista cumplen una condición\par
\subsection*{Ejemplos}
\begin{quote}
{\haddockverb\begin{verbatim}
>>> all' (<10) [1,2,3,4]
True

\end{verbatim}}
\end{quote}
\begin{quote}
{\haddockverb\begin{verbatim}
>>> all' (>10) [1,20,30,40]
False

\end{verbatim}}
\end{quote}}
\end{haddockdesc}
\subsection{D)}
\begin{haddockdesc}
\item[\begin{tabular}{@{}l}
min' :: Ord a => {\char 91}a{\char 93} -> a
\end{tabular}]
{\haddockbegindoc
\section*{Descripción}
Devuelve el menor elemento de una lista dada\par
\subsection*{Ejemplos}
\begin{quote}
{\haddockverb\begin{verbatim}
>>> min' []
Lista vacia

\end{verbatim}}
\end{quote}
\begin{quote}
{\haddockverb\begin{verbatim}
>>> min' [2,5,8,2,0]
0

\end{verbatim}}
\end{quote}}
\end{haddockdesc}
\subsection{E)}
\begin{haddockdesc}
\item[\begin{tabular}{@{}l}
map' :: (a -> b) -> {\char 91}a{\char 93} -> {\char 91}b{\char 93}
\end{tabular}]
{\haddockbegindoc
\section*{Descripción}
Genera una lista de elementos resultantes de aplicar un filtro a otra lista de elementos\par
\subsection*{Ejemplo}
\begin{quote}
{\haddockverb\begin{verbatim}
>>> map' (*3) [1,2,3,4]
[3,6,9,12]

\end{verbatim}}
\end{quote}}
\end{haddockdesc}
\subsection{F)}
\begin{haddockdesc}
\item[\begin{tabular}{@{}l}
filter' :: (a -> Bool) -> {\char 91}a{\char 93} -> {\char 91}a{\char 93}
\end{tabular}]
{\haddockbegindoc
\section*{Descripción}
Genera una lista de elementos con los elementos de otra lista que hayan pasado un filtro\par
\subsection*{Ejemplo}
\begin{quote}
{\haddockverb\begin{verbatim}
>>> filter' (>10) [1,20,30,40]
[20,30,40]

\end{verbatim}}
\end{quote}}
\end{haddockdesc}
\subsection{G)}
\begin{haddockdesc}
\item[\begin{tabular}{@{}l}
takeWhile' :: (a -> Bool) -> {\char 91}a{\char 93} -> {\char 91}a{\char 93}
\end{tabular}]
{\haddockbegindoc
\section*{Descripción}
Coge elementos de la lista dada hasta que la condición falla\par
\subsection*{Ejemplos}
\begin{quote}
{\haddockverb\begin{verbatim}
>>> takeWhile' (<3) [1,2,3,4,5]
[1,2]

\end{verbatim}}
\end{quote}
\begin{quote}
{\haddockverb\begin{verbatim}
>>> takeWhile' (>3) [1,2,3,4,5]
[]

\end{verbatim}}
\end{quote}}
\end{haddockdesc}
\section{EJERCICIO 2}
\begin{haddockdesc}
\item[\begin{tabular}{@{}l}
listaNegPos :: {\char 91}Integer{\char 93}
\end{tabular}]
{\haddockbegindoc
\section*{Descripción}
Genera una lista alternada de elementos del 1 al 100 de forma {\char 91}1, -1{\char 93}\par}
\end{haddockdesc}
\section{EJERCICIO 3}
\begin{haddockdesc}
\item[\begin{tabular}{@{}l}
listaParejas' :: (Num b, Enum b, Eq b) => b -> {\char 91}(b, b){\char 93}
\end{tabular}]
{\haddockbegindoc
\section*{Descripción}
Versión acotada de \haddockid{listaParejas}\par
\subsection*{Ejemplo}
\begin{quote}
{\haddockverb\begin{verbatim}
>>> listaParejas' 4
[(0,0),(0,1),(1,0),(0,2),(1,1),(2,0),(0,3),(1,2),(2,1),(3,0),(0,4),(1,3),(2,2),(3,1),(4,0)]

\end{verbatim}}
\end{quote}}
\end{haddockdesc}
\begin{haddockdesc}
\item[\begin{tabular}{@{}l}
listaParejas :: Integral a => {\char 91}(a, a){\char 93}
\end{tabular}]
{\haddockbegindoc
\section*{Descripción}
Genera parejas ordenadas que cumplen x + y = z, en orden de z\par
Ver \haddockid{listaParejas'} para ejemplos\par}
\end{haddockdesc}
\section{EJERCICIO 4}
\subsection{A)}
\begin{haddockdesc}
\item[\begin{tabular}{@{}l}
sufijos :: {\char 91}a{\char 93} -> {\char 91}{\char 91}a{\char 93}{\char 93}
\end{tabular}]
{\haddockbegindoc
\section*{Descripción}
Genera todos los sufijos de una lista\par
\subsection*{Ejemplo}
\begin{quote}
{\haddockverb\begin{verbatim}
>>> sufijos [1,2,3,4]
[[1,2,3,4],[2,3,4],[3,4],[4],[]]

\end{verbatim}}
\end{quote}}
\end{haddockdesc}
\subsection{B)}
\begin{haddockdesc}
\item[\begin{tabular}{@{}l}
sublistas :: {\char 91}a{\char 93} -> {\char 91}{\char 91}a{\char 93}{\char 93}
\end{tabular}]
{\haddockbegindoc
\section*{Descripción}
Genera todas las sublistas de una lista dada\par
\subsection*{Ejemplo}
\begin{quote}
{\haddockverb\begin{verbatim}
>>> sublistas [1,2,3]
[[1],[2],[3],[1,2],[2,3],[1,2,3]]

\end{verbatim}}
\end{quote}}
\end{haddockdesc}
\subsection{C)}
\begin{haddockdesc}
\item[\begin{tabular}{@{}l}
permuta :: {\char 91}a{\char 93} -> {\char 91}{\char 91}a{\char 93}{\char 93}
\end{tabular}]
{\haddockbegindoc
\section*{Descripción}
Genera todas las permutaciones posibles de una lista dada\par
\subsection*{Ejemplo}
\begin{quote}
{\haddockverb\begin{verbatim}
>>> permuta [1,2,3]
[[1,2,3],[2,1,3],[2,3,1],[1,3,2],[3,1,2],[3,2,1]]

\end{verbatim}}
\end{quote}}
\end{haddockdesc}
\begin{haddockdesc}
\item[\begin{tabular}{@{}l}
intercalado :: a -> {\char 91}a{\char 93} -> {\char 91}{\char 91}a{\char 93}{\char 93}
\end{tabular}]
{\haddockbegindoc
\section*{Descripción}
Devuelve una lista de listas con el elemento dado en cada posición posible de la lista dada\par
\subsection*{Ejemplo}
\begin{quote}
{\haddockverb\begin{verbatim}
>>> intercalado 3 [1,2]
[[3,1,2],[1,3,2],[1,2,3]]

\end{verbatim}}
\end{quote}}
\end{haddockdesc}
\subsection{D)}
\begin{haddockdesc}
\item[\begin{tabular}{@{}l}
sumandos :: Integral a => a -> {\char 91}{\char 91}a{\char 93}{\char 93}
\end{tabular}]
{\haddockbegindoc
\section*{Descripción}
Genera un listado de de listas que al sumar todos sus elementos dan como resultado el numero dado\par
\subsection*{Ejemplos}
\begin{quote}
{\haddockverb\begin{verbatim}
>>> sumandos 3
[[1,1,1],[1,2],[3]]

\end{verbatim}}
\end{quote}}
\end{haddockdesc}
\begin{haddockdesc}
\item[\begin{tabular}{@{}l}
sumandos' :: Integral a => a -> a -> a -> {\char 91}{\char 91}a{\char 93}{\char 93}
\end{tabular}]
{\haddockbegindoc
\section*{Descripción}
Genera listados de tamaños desde x hasta n que al ser sumados dan como resultado x, siendo i el numero de listados generados\par
\subsection*{Ejemplo}
\begin{quote}
{\haddockverb\begin{verbatim}
>>> sumandos' 4 1 1
[[1,1,1],[1,2],[3]]

\end{verbatim}}
\end{quote}
\begin{quote}
{\haddockverb\begin{verbatim}
>>> sumandos' 4 2 1
[[1,1],[2]]

\end{verbatim}}
\end{quote}
\begin{quote}
{\haddockverb\begin{verbatim}
>>> sumandos' 4 2 2
[[2]]

\end{verbatim}}
\end{quote}}
\end{haddockdesc}