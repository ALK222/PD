\haddockmoduleheading{Labo5}
\label{module:Labo5}
\haddockbeginheader
{\haddockverb\begin{verbatim}
module Labo5 (
    getInt, adivina, Matriz, getFloat, getMatriz, getFilas, getDatosMatriz,
    dibujaMatriz, mostrarMatrizNueva, formatea, formatear, justify, addSpaces
  ) where\end{verbatim}}
\haddockendheader

\section{EJERCICIO 1}
\begin{haddockdesc}
\item[\begin{tabular}{@{}l}
getInt :: IO Int
\end{tabular}]
{\haddockbegindoc
\section*{Descripción}
Coge un número desde la entrada.\par}
\end{haddockdesc}
\begin{haddockdesc}
\item[\begin{tabular}{@{}l}
adivina :: Int -> IO ()
\end{tabular}]
{\haddockbegindoc
\section*{Descripión}
Pide un número por teclado a adivinar. Esto es bastante estúpido de por si solo porque tienes que introducir el número como argumento\par}
\end{haddockdesc}
\section{EJERCICIO 2}
\begin{haddockdesc}
\item[\begin{tabular}{@{}l}
type Matriz = {\char 91}{\char 91}Float{\char 93}{\char 93}
\end{tabular}]
{\haddockbegindoc
Representación de una matriz como lista de listas\par}
\end{haddockdesc}
\begin{haddockdesc}
\item[\begin{tabular}{@{}l}
getFloat :: IO Float
\end{tabular}]
{\haddockbegindoc
\section*{Descripción}
Coge un número desde la entrada.\par}
\end{haddockdesc}
\subsection{A)}
\begin{haddockdesc}
\item[\begin{tabular}{@{}l}
getMatriz :: IO Matriz
\end{tabular}]
{\haddockbegindoc
\section*{Descripción}
Coge una matriz desde la entrada.\par
Ver \haddockid{getDatosMatriz} para ver el funcionamiento total\par}
\end{haddockdesc}
\begin{haddockdesc}
\item[\begin{tabular}{@{}l}
getFilas :: Int -> IO {\char 91}Float{\char 93}
\end{tabular}]
{\haddockbegindoc
\section*{Descripción}
Coge una fila de una matriz según el número de columnas dadas.\par}
\end{haddockdesc}
\begin{haddockdesc}
\item[\begin{tabular}{@{}l}
getDatosMatriz :: Int -> Int -> IO Matriz
\end{tabular}]
{\haddockbegindoc
\section*{Descripción}
Coge todas las filas de una matriz según el número de filas y columnas dadas.\par
Ver \haddockid{getFilas} para el funcionamiento completo\par}
\end{haddockdesc}
\subsection{B)}
\begin{haddockdesc}
\item[\begin{tabular}{@{}l}
dibujaMatriz :: Matriz -> IO ()
\end{tabular}]
{\haddockbegindoc
\section*{Descripción}
Dibuja una matriz dada como argumento.\par}
\end{haddockdesc}
\begin{haddockdesc}
\item[\begin{tabular}{@{}l}
mostrarMatrizNueva :: IO ()
\end{tabular}]
{\haddockbegindoc
\section*{Descripción}
Pide una matriz por teclado y la muestra una vez construida.\par
Ver \haddockid{getMatriz} y \haddockid{dibujaMatriz}.\par}
\end{haddockdesc}
\section{EJERCICIO 3}
\begin{haddockdesc}
\item[\begin{tabular}{@{}l}
formatea :: String -> String -> Int -> IO ()
\end{tabular}]
{\haddockbegindoc
\section*{Descripción}
Formatea a n columnas cada linea de un fichero dado en otro.\par
Ver \haddockid{formatear} para el funcionamiento.\par}
\end{haddockdesc}
\begin{haddockdesc}
\item[\begin{tabular}{@{}l}
formatear :: Int -> String -> String
\end{tabular}]
{\haddockbegindoc
\section*{Descripción}
Formatea cada linea a un total de n caracteres por fila.\par
Ver \haddockid{justify} para el funcionamiento completo.\par}
\end{haddockdesc}
\begin{haddockdesc}
\item[\begin{tabular}{@{}l}
justify :: Int -> String -> String
\end{tabular}]
{\haddockbegindoc
\section*{Descripción}
Añade espacios a una línea si es necesario.\par
Ver \haddockid{addSpaces} para el funcionamiento completo.\par}
\end{haddockdesc}
\begin{haddockdesc}
\item[\begin{tabular}{@{}l}
addSpaces :: Int -> String -> String
\end{tabular}]
{\haddockbegindoc
\section*{Descripción}
Añade n espacios a una lina dada.\par}
\end{haddockdesc}