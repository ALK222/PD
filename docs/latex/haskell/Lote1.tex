\haddockmoduleheading{Lote1}
\label{module:Lote1}
\haddockbeginheader
{\haddockverb\begin{verbatim}
module Lote1 (
    petaFlop, fornl, yearsToSec, frontierInteger, petaFlop', fornl',
    yearsToSec', frontierInt, petaFlop'', fornl'', yearsToSec'', frontierDouble,
    time, totalYears, daysToSec, hoursToSec, minsToSec, years, days, hours,
    mins, segs, solc, sold, f, f', g, h, h', i, i', digitos, sumaDigitos,
    reduccion, perm, var, comb, factores, isPrime, fibonacci, y, o, no, implica,
    eq, oEx, (&&&), (|||), (-->), (===), (||=), f8, media, last', init',
    initLast, concat', auxList, take', drop', splitAt', nub, quicksort, and',
    or', sum', product', lmedia, reverse', reverse''
  ) where\end{verbatim}}
\haddockendheader

\section{EJERCICIO 1}
\subsection{A)}
\begin{haddockdesc}
\item[\begin{tabular}{@{}l}
petaFlop :: Integer
\end{tabular}]
{\haddockbegindoc
Flops in a petaFlop\par}
\end{haddockdesc}
\begin{haddockdesc}
\item[\begin{tabular}{@{}l}
fornl :: Integer
\end{tabular}]
{\haddockbegindoc
Computing power of the best computer of Frontier\par}
\end{haddockdesc}
\begin{haddockdesc}
\item[\begin{tabular}{@{}l}
yearsToSec :: Integer
\end{tabular}]
{\haddockbegindoc
Seconds in a year\par}
\end{haddockdesc}
\begin{haddockdesc}
\item[\begin{tabular}{@{}l}
frontierInteger :: Integer
\end{tabular}]
{\haddockbegindoc
Power of operation of the Frontier since the beginning of time
 | >>> frontierInteger
 515859580800000000000000000000000000\par}
\end{haddockdesc}
\begin{haddockdesc}
\item[\begin{tabular}{@{}l}
petaFlop' :: Int
\end{tabular}]
{\haddockbegindoc
Flops in a petaFlop\par}
\end{haddockdesc}
\begin{haddockdesc}
\item[\begin{tabular}{@{}l}
fornl' :: Int
\end{tabular}]
{\haddockbegindoc
Computing power of the best computer of Frontier\par}
\end{haddockdesc}
\begin{haddockdesc}
\item[\begin{tabular}{@{}l}
yearsToSec' :: Int
\end{tabular}]
{\haddockbegindoc
Seconds in a year\par}
\end{haddockdesc}
\begin{haddockdesc}
\item[\begin{tabular}{@{}l}
frontierInt :: Int
\end{tabular}]
{\haddockbegindoc
Power of operation of the Frontier since the beginning of time
 | >>> frontierInt
 -7928755680649936896\par}
\end{haddockdesc}
\begin{haddockdesc}
\item[\begin{tabular}{@{}l}
petaFlop'' :: Double
\end{tabular}]
{\haddockbegindoc
Flops in a petaFlop\par}
\end{haddockdesc}
\begin{haddockdesc}
\item[\begin{tabular}{@{}l}
fornl'' :: Double
\end{tabular}]
{\haddockbegindoc
Computing power of the best computer of Frontier\par}
\end{haddockdesc}
\begin{haddockdesc}
\item[\begin{tabular}{@{}l}
yearsToSec'' :: Double
\end{tabular}]
{\haddockbegindoc
Seconds in a year\par}
\end{haddockdesc}
\begin{haddockdesc}
\item[\begin{tabular}{@{}l}
frontierDouble :: Double
\end{tabular}]
{\haddockbegindoc
Power of operation of the Frontier since the beginning of time
 | >>> frontierDouble
 5.158595808e35\par}
\end{haddockdesc}
\subsection{B)}
\begin{haddockdesc}
\item[\begin{tabular}{@{}l}
time :: Integer
\end{tabular}]
{\haddockbegindoc
Time for the exercise\par}
\end{haddockdesc}
\begin{haddockdesc}
\item[\begin{tabular}{@{}l}
totalYears :: Integer
\end{tabular}]
{\haddockbegindoc
\begin{quote}
{\haddockverb\begin{verbatim}
>>> totalYears
317

\end{verbatim}}
\end{quote}}
\end{haddockdesc}
\subsection{C)}
\begin{haddockdesc}
\item[\begin{tabular}{@{}l}
daysToSec :: Integer
\end{tabular}]
{\haddockbegindoc
Seconds in a day\par}
\end{haddockdesc}
\begin{haddockdesc}
\item[\begin{tabular}{@{}l}
hoursToSec :: Integer
\end{tabular}]
{\haddockbegindoc
Seconds in an hour\par}
\end{haddockdesc}
\begin{haddockdesc}
\item[\begin{tabular}{@{}l}
minsToSec :: Integer
\end{tabular}]
{\haddockbegindoc
Segundos totales en un minuto\par}
\end{haddockdesc}
\begin{haddockdesc}
\item[\begin{tabular}{@{}l}
years :: Integer
\end{tabular}]
{\haddockbegindoc
Caclulo de los años equivalentes a \haddockid{time}\par}
\end{haddockdesc}
\begin{haddockdesc}
\item[\begin{tabular}{@{}l}
days :: Integer
\end{tabular}]
{\haddockbegindoc
Calculo de las horas equivalentes a \haddockid{time} quitando \haddockid{years}\par}
\end{haddockdesc}
\begin{haddockdesc}
\item[\begin{tabular}{@{}l}
hours :: Integer
\end{tabular}]
{\haddockbegindoc
Calculo de las horas equivalentes a \haddockid{time} quitando \haddockid{days} y \haddockid{years}\par}
\end{haddockdesc}
\begin{haddockdesc}
\item[\begin{tabular}{@{}l}
mins :: Integer
\end{tabular}]
{\haddockbegindoc
Calculo de las horas equivalentes a \haddockid{time} quitando \haddockid{days}, \haddockid{years} y \haddockid{hours}\par}
\end{haddockdesc}
\begin{haddockdesc}
\item[\begin{tabular}{@{}l}
segs :: Integer
\end{tabular}]
{\haddockbegindoc
Calculo de las horas equivalentes a \haddockid{time} quitando \haddockid{days}, \haddockid{years}, \haddockid{hours} y \haddockid{mins}\par}
\end{haddockdesc}
\begin{haddockdesc}
\item[\begin{tabular}{@{}l}
solc :: (Integer, Integer, Integer, Integer, Integer)
\end{tabular}]
{\haddockbegindoc
Tupla con el tiempo total de \haddockid{time} en años, dias, horas, minutos y segundos\par}
\end{haddockdesc}
\subsection{D)}
\begin{haddockdesc}
\item[\begin{tabular}{@{}l}
sold :: Integer -> (Integer, Integer, Integer, Integer, Integer)
\end{tabular}]
{\haddockbegindoc
\section*{Descripción}
Coge el tiempo dado y lo convierte a (Años, Días, Horas, Minutos, Segundos)\par
\subsection*{Ejemplos}
\begin{quote}
{\haddockverb\begin{verbatim}
>>> sold 1000000
(0,11,13,46,40)

\end{verbatim}}
\end{quote}}
\end{haddockdesc}
\section{EJERCICIO 2}
\begin{haddockdesc}
\item[\begin{tabular}{@{}l}
f :: Int -> Int -> Int
\end{tabular}]
{\haddockbegindoc
\section*{Descripción}
Función f usando notación infija\par
\subsection*{Ejemplos}
\begin{quote}
{\haddockverb\begin{verbatim}
>>> f 7 4
-14

\end{verbatim}}
\end{quote}}
\end{haddockdesc}
\begin{haddockdesc}
\item[\begin{tabular}{@{}l}
f' :: Int -> Int -> Int
\end{tabular}]
{\haddockbegindoc
\section*{Descripción}
Función f usando notación prefija\par
\subsection*{Ejemplo}
\begin{quote}
{\haddockverb\begin{verbatim}
>>> f' 7 4
-14

\end{verbatim}}
\end{quote}}
\end{haddockdesc}
\begin{haddockdesc}
\item[\begin{tabular}{@{}l}
g :: Int -> Int
\end{tabular}]
{\haddockbegindoc
\section*{Descripción}
Función g\par
\subsection*{Ejemplo}
\begin{quote}
{\haddockverb\begin{verbatim}
>>> g (-2)
32

\end{verbatim}}
\end{quote}}
\end{haddockdesc}
\begin{haddockdesc}
\item[\begin{tabular}{@{}l}
h :: Int -> Int -> Int -> Int
\end{tabular}]
{\haddockbegindoc
\section*{Descripción}
Función h usando notación infija\par
\subsection*{Ejemplo}
\begin{quote}
{\haddockverb\begin{verbatim}
>>> h 1 2 3
0

\end{verbatim}}
\end{quote}}
\end{haddockdesc}
\begin{haddockdesc}
\item[\begin{tabular}{@{}l}
h' :: Int -> Int -> Int -> Int
\end{tabular}]
{\haddockbegindoc
\section*{Descripción}
Función h usando notación posfija\par
\subsection*{Ejemplo}
\begin{quote}
{\haddockverb\begin{verbatim}
>>> h' 1 2 3
0

\end{verbatim}}
\end{quote}}
\end{haddockdesc}
\begin{haddockdesc}
\item[\begin{tabular}{@{}l}
i :: Int -> Int -> Int
\end{tabular}]
{\haddockbegindoc
\section*{Descripción}
Función i usando notación guardas\par
\subsection*{Ejemplo}
\begin{quote}
{\haddockverb\begin{verbatim}
>>> i 1 2
0

\end{verbatim}}
\end{quote}}
\end{haddockdesc}
\begin{haddockdesc}
\item[\begin{tabular}{@{}l}
i' :: Int -> Int -> Int
\end{tabular}]
{\haddockbegindoc
\section*{Descripción}
Función i usando notación if {\char '137} then {\char '137} else\par
\subsection*{Ejemplo}
\begin{quote}
{\haddockverb\begin{verbatim}
>>> i 1 2
0

\end{verbatim}}
\end{quote}}
\end{haddockdesc}
\section{EJERCICIO 3}
\subsection{A)}
\begin{haddockdesc}
\item[\begin{tabular}{@{}l}
digitos :: Int -> Int
\end{tabular}]
{\haddockbegindoc
\section*{Description}
Gets the number of digits of a given value\par
\subsection*{Example}
\begin{quote}
{\haddockverb\begin{verbatim}
>>> digitos 1
1

\end{verbatim}}
\end{quote}
\begin{quote}
{\haddockverb\begin{verbatim}
>>> digitos 123912891289
12

\end{verbatim}}
\end{quote}}
\end{haddockdesc}
\subsection{B)}
\begin{haddockdesc}
\item[\begin{tabular}{@{}l}
sumaDigitos :: Int -> Int
\end{tabular}]
{\haddockbegindoc
\section*{Description}
Adds every single digit of a given number\par
\subsection*{Example}
\begin{quote}
{\haddockverb\begin{verbatim}
>>> sumaDigitos 1
1

\end{verbatim}}
\end{quote}
\begin{quote}
{\haddockverb\begin{verbatim}
>>> sumaDigitos 18982918192283429138194213189
82

\end{verbatim}}
\end{quote}}
\end{haddockdesc}
\begin{haddockdesc}
\item[\begin{tabular}{@{}l}
reduccion :: Int -> Int
\end{tabular}]
{\haddockbegindoc
\section*{Description}
Adds digits (\haddockid{sumaDigitos}) from a number until de result is one single digit. Single digits bellow 0 are converted to positive values\par
\subsection*{Example}
\begin{quote}
{\haddockverb\begin{verbatim}
>>> reduccion (-1)
1

\end{verbatim}}
\end{quote}
\begin{quote}
{\haddockverb\begin{verbatim}
>>> reduccion 12893213913819238192312819
2

\end{verbatim}}
\end{quote}}
\end{haddockdesc}
\subsection{C)}
\begin{haddockdesc}
\item[\begin{tabular}{@{}l}
perm :: Integer -> Integer
\end{tabular}]
{\haddockbegindoc
\section*{Description}
Permutations of a given number\par
\subsection*{Example}
\begin{quote}
{\haddockverb\begin{verbatim}
>>> perm 8
40320

\end{verbatim}}
\end{quote}}
\end{haddockdesc}
\subsection{D)}
\begin{haddockdesc}
\item[\begin{tabular}{@{}l}
var :: Integer -> Integer -> Integer
\end{tabular}]
{\haddockbegindoc
\section*{Description}
Variations of two given numbers\par
\subsection*{Example}
\begin{quote}
{\haddockverb\begin{verbatim}
>>> var 2 1
2

\end{verbatim}}
\end{quote}
\begin{quote}
{\haddockverb\begin{verbatim}
>>> var 123 5
25928567280

\end{verbatim}}
\end{quote}}
\end{haddockdesc}
\subsection{E)}
\begin{haddockdesc}
\item[\begin{tabular}{@{}l}
comb :: Integer -> Integer -> Integer
\end{tabular}]
{\haddockbegindoc
\section*{Description}
Combinations of two given numbers\par
\subsection*{Examples}
\begin{quote}
{\haddockverb\begin{verbatim}
>>> comb 2 1
2

\end{verbatim}}
\end{quote}
\begin{quote}
{\haddockverb\begin{verbatim}
>>> comb 12425 12423
77184100

\end{verbatim}}
\end{quote}}
\end{haddockdesc}
\section{EJERCICIO 4}
\begin{haddockdesc}
\item[\begin{tabular}{@{}l}
factores :: Integral a => a -> {\char 91}a{\char 93}
\end{tabular}]
{\haddockbegindoc
\section*{Descripción}
Calcula todos los divisores de un número dado\par
\subsection*{Ejemplos}
\begin{quote}
{\haddockverb\begin{verbatim}
>>> factores 10
[1,2,5,10]

\end{verbatim}}
\end{quote}}
\end{haddockdesc}
\begin{haddockdesc}
\item[\begin{tabular}{@{}l}
isPrime :: Integral a => a -> Bool
\end{tabular}]
{\haddockbegindoc
\section*{Descripción}
Comprueba si un número es primo\par
Usa \haddockid{factores}\par
\subsection*{Ejemplos}
\begin{quote}
{\haddockverb\begin{verbatim}
>>> isPrime 10
False

\end{verbatim}}
\end{quote}
\begin{quote}
{\haddockverb\begin{verbatim}
>>> isPrime 5
True

\end{verbatim}}
\end{quote}}
\end{haddockdesc}
\section{EJERCICIO 5}
\begin{haddockdesc}
\item[\begin{tabular}{@{}l}
fibonacci :: Int -> Integer
\end{tabular}]
{\haddockbegindoc
\section*{Descipción}
Funcion de fibonacci de aridad 1\par
\subsection*{Ejemplos}
\begin{quote}
{\haddockverb\begin{verbatim}
>>> fibonacci 2
2

\end{verbatim}}
\end{quote}
\begin{quote}
{\haddockverb\begin{verbatim}
>>> fibonacci 100
573147844013817084101

\end{verbatim}}
\end{quote}}
\end{haddockdesc}
\section{EJERCICIO 6}
\begin{haddockdesc}
\item[\begin{tabular}{@{}l}
y :: Bool -> Bool -> Bool
\end{tabular}]
{\haddockbegindoc
Conjunction operator\par}
\end{haddockdesc}
\begin{haddockdesc}
\item[\begin{tabular}{@{}l}
o :: Bool -> Bool -> Bool
\end{tabular}]
{\haddockbegindoc
Disjunction operator\par}
\end{haddockdesc}
\begin{haddockdesc}
\item[\begin{tabular}{@{}l}
no :: Bool -> Bool
\end{tabular}]
{\haddockbegindoc
Not operator\par}
\end{haddockdesc}
\begin{haddockdesc}
\item[\begin{tabular}{@{}l}
implica :: Bool -> Bool -> Bool
\end{tabular}]
{\haddockbegindoc
Implication operator\par}
\end{haddockdesc}
\begin{haddockdesc}
\item[\begin{tabular}{@{}l}
eq :: Bool -> Bool -> Bool
\end{tabular}]
{\haddockbegindoc
Equivalence operator\par}
\end{haddockdesc}
\begin{haddockdesc}
\item[\begin{tabular}{@{}l}
oEx :: Bool -> Bool -> Bool
\end{tabular}]
{\haddockbegindoc
Exclusive disjunction operator\par}
\end{haddockdesc}
\begin{haddockdesc}
\item[\begin{tabular}{@{}l}
({\char '46}{\char '46}{\char '46}) :: Bool -> Bool -> Bool
\end{tabular}]
{\haddockbegindoc
Conjunction operator\par}
\end{haddockdesc}
\begin{haddockdesc}
\item[\begin{tabular}{@{}l}
(|||) :: Bool -> Bool -> Bool
\end{tabular}]
{\haddockbegindoc
Disjunction operator\par}
\end{haddockdesc}
\begin{haddockdesc}
\item[\begin{tabular}{@{}l}
(-->) :: Bool -> Bool -> Bool
\end{tabular}]
{\haddockbegindoc
Implication operator\par}
\end{haddockdesc}
\begin{haddockdesc}
\item[\begin{tabular}{@{}l}
(===) :: Bool -> Bool -> Bool
\end{tabular}]
{\haddockbegindoc
Equivalence operator\par}
\end{haddockdesc}
\begin{haddockdesc}
\item[\begin{tabular}{@{}l}
(||=) :: Bool -> Bool -> Bool
\end{tabular}]
{\haddockbegindoc
Exclusive disjunction operator\par}
\end{haddockdesc}
\section{EJERCICIO 8}
\begin{haddockdesc}
\item[\begin{tabular}{@{}l}
f8 :: Int -> Int -> Int -> Int
\end{tabular}]
{\haddockbegindoc
\section*{Description}
La función toma tres argumentos y cumple con las siguientes condiciones:\par
(i) Ser estricta en el primer argumento.\par
(ii) No ser estricta ni en el segundo ni en el tercer argumento.\par
(iii) Ser conjuntamente estricta en el segundo y tercer argumento.\par
\subsection*{Examples}
\begin{quote}
{\haddockverb\begin{verbatim}
>>> f8 1 undefined undefined
1

\end{verbatim}}
\end{quote}
\begin{quote}
{\haddockverb\begin{verbatim}
>>> f8 0 1 2
3

\end{verbatim}}
\end{quote}
\begin{quote}
{\haddockverb\begin{verbatim}
>>> f8 0 1 undefined
Prelude.undefined

\end{verbatim}}
\end{quote}}
\end{haddockdesc}
\section{EJERCICIO 9}
\begin{haddockdesc}
\item[\begin{tabular}{@{}l}
media :: (Fractional a, Integral a, Foldable t) => t a -> a
\end{tabular}]
{\haddockbegindoc
\section*{Description}
Arithmetic mean of a list of numbers\par
\subsection*{Example}
\begin{quote}
{\haddockverb\begin{verbatim}
>>> media [1,2,3,4]
2.5

\end{verbatim}}
\end{quote}}
\end{haddockdesc}
\section{EJERCICIO 10}
\begin{haddockdesc}
\item[\begin{tabular}{@{}l}
last' :: {\char 91}a{\char 93} -> a
\end{tabular}]
{\haddockbegindoc
\section*{Description}
Returns the last element of a given list\par
\subsection*{Example}
\begin{quote}
{\haddockverb\begin{verbatim}
>>> last' [1,2,3,4]
4

\end{verbatim}}
\end{quote}}
\end{haddockdesc}
\begin{haddockdesc}
\item[\begin{tabular}{@{}l}
init' :: {\char 91}a{\char 93} -> {\char 91}a{\char 93}
\end{tabular}]
{\haddockbegindoc
\section*{Description}
Returns all element except the las one of a list\par
\subsection*{Examples}
\begin{quote}
{\haddockverb\begin{verbatim}
>>> init [1,2,3,4]
[1,2,3]

\end{verbatim}}
\end{quote}}
\end{haddockdesc}
\begin{haddockdesc}
\item[\begin{tabular}{@{}l}
initLast :: {\char 91}a{\char 93} -> ({\char 91}a{\char 93}, a)
\end{tabular}]
{\haddockbegindoc
\section*{Description}
Returns a tuple with the \haddockid{init'} of a list and the \haddockid{last'} of a list\par
\subsection*{Example}
\begin{quote}
{\haddockverb\begin{verbatim}
>>> initLast [1,2,3,4]
([1,2,3],4)

\end{verbatim}}
\end{quote}}
\end{haddockdesc}
\begin{haddockdesc}
\item[\begin{tabular}{@{}l}
concat' :: {\char 91}{\char 91}a{\char 93}{\char 93} -> {\char 91}a{\char 93}
\end{tabular}]
{\haddockbegindoc
\section*{Description}
Concatenates all given lists into one\par
\subsection*{Example}
\begin{quote}
{\haddockverb\begin{verbatim}
>>> concat' [[1],[2,3],[4],[5]]
[1,2,3,4,5]

\end{verbatim}}
\end{quote}}
\end{haddockdesc}
\begin{haddockdesc}
\item[\begin{tabular}{@{}l}
auxList :: Int -> Int -> Int -> {\char 91}a{\char 93} -> {\char 91}a{\char 93}
\end{tabular}]
{\haddockbegindoc
\section*{Description}
Returns the elements of a list between two given indexes\par
\subsection*{Examples}
\begin{quote}
{\haddockverb\begin{verbatim}
>>> auxList 0 0 2 [1,2,3,4,5]
[1,2,3]

\end{verbatim}}
\end{quote}}
\end{haddockdesc}
\begin{haddockdesc}
\item[\begin{tabular}{@{}l}
take' :: Int -> {\char 91}a{\char 93} -> {\char 91}a{\char 93}
\end{tabular}]
{\haddockbegindoc
\section*{Description}
Takes the n first elements of a list\par
\subsection*{Example}
\begin{quote}
{\haddockverb\begin{verbatim}
>>> take' 2 [1,2,3,4]
[1,2]

\end{verbatim}}
\end{quote}}
\end{haddockdesc}
\begin{haddockdesc}
\item[\begin{tabular}{@{}l}
drop' :: Int -> {\char 91}a{\char 93} -> {\char 91}a{\char 93}
\end{tabular}]
{\haddockbegindoc
\section*{Description}
Erases the n first elements of a list\par
\subsection*{Example}
\begin{quote}
{\haddockverb\begin{verbatim}
>>> drop' 2 [1,2,3,4]
[3,4]

\end{verbatim}}
\end{quote}}
\end{haddockdesc}
\begin{haddockdesc}
\item[\begin{tabular}{@{}l}
splitAt' :: Int -> {\char 91}a{\char 93} -> ({\char 91}a{\char 93}, {\char 91}a{\char 93})
\end{tabular}]
{\haddockbegindoc
\section*{Description}
Splits a list in the given index\par
\subsection*{Example}
\begin{quote}
{\haddockverb\begin{verbatim}
>>> splitAt' 2 [1,2,3,4]
([1,2],[3,4])

\end{verbatim}}
\end{quote}}
\end{haddockdesc}
\begin{haddockdesc}
\item[\begin{tabular}{@{}l}
nub :: Eq a => {\char 91}a{\char 93} -> {\char 91}a{\char 93}
\end{tabular}]
{\haddockbegindoc
\section*{Description}
Returns a list without the repeated values of the given list\par
\subsection*{Example}
\begin{quote}
{\haddockverb\begin{verbatim}
>>> nub [1,2,3,4,1,2,3,4,3,2,6,4,2,0]
[1,2,3,4,6,0]

\end{verbatim}}
\end{quote}}
\end{haddockdesc}
\begin{haddockdesc}
\item[\begin{tabular}{@{}l}
quicksort :: Ord a => {\char 91}a{\char 93} -> {\char 91}a{\char 93}
\end{tabular}]
{\haddockbegindoc
\section*{Description}
Quicksort implementation\par
\subsection*{Examples}
\begin{quote}
{\haddockverb\begin{verbatim}
>>> quicksort [4,7,3,6,1,89,52,76,13,0,23]
[0,1,3,4,6,7,13,23,52,76,89]

\end{verbatim}}
\end{quote}}
\end{haddockdesc}
\begin{haddockdesc}
\item[\begin{tabular}{@{}l}
and' :: {\char 91}Bool{\char 93} -> Bool
\end{tabular}]
{\haddockbegindoc
\section*{Description}
Conjunction of al elements of a list\par
\subsection*{Example}
\begin{quote}
{\haddockverb\begin{verbatim}
>>> and' [True,False, True ]
False

\end{verbatim}}
\end{quote}}
\end{haddockdesc}
\begin{haddockdesc}
\item[\begin{tabular}{@{}l}
or' :: {\char 91}Bool{\char 93} -> Bool
\end{tabular}]
{\haddockbegindoc
\section*{Description}
Disjunction of al elements of a list\par
\subsection*{Example}
\begin{quote}
{\haddockverb\begin{verbatim}
>>> or' [True,False, True ]
True

\end{verbatim}}
\end{quote}}
\end{haddockdesc}
\begin{haddockdesc}
\item[\begin{tabular}{@{}l}
sum' :: {\char 91}Int{\char 93} -> Int
\end{tabular}]
{\haddockbegindoc
\section*{Description}
Sum of al the numbers in the list\par
\subsection*{Example}
\begin{quote}
{\haddockverb\begin{verbatim}
>>> sum' [1..100]
5050

\end{verbatim}}
\end{quote}}
\end{haddockdesc}
\begin{haddockdesc}
\item[\begin{tabular}{@{}l}
product' :: {\char 91}Integer{\char 93} -> Integer
\end{tabular}]
{\haddockbegindoc
\section*{Description}
Product of al the numbers in the list\par
\subsection*{Example}
\begin{quote}
{\haddockverb\begin{verbatim}
>>> product' [2..5]
120

\end{verbatim}}
\end{quote}}
\end{haddockdesc}
\begin{haddockdesc}
\item[\begin{tabular}{@{}l}
lmedia :: {\char 91}{\char 91}{\char 91}a{\char 93}{\char 93}{\char 93} -> Float
\end{tabular}]
{\haddockbegindoc
\section*{Description}
Mean of sizes of al elements in a list of lists\par}
\end{haddockdesc}
\section{EJERCICIO 11}
\subsection{A)}
\begin{haddockdesc}
\item[\begin{tabular}{@{}l}
reverse' :: {\char 91}a{\char 93} -> {\char 91}a{\char 93}
\end{tabular}]
{\haddockbegindoc
\section*{Description}
Reverse a given list in o(n{\char '136}2) time\par
\subsection*{Example}
\begin{quote}
{\haddockverb\begin{verbatim}
>>> reverse' [1,2,3,4]
[4,3,2,1]

\end{verbatim}}
\end{quote}}
\end{haddockdesc}
\subsection{B)}
\begin{haddockdesc}
\item[\begin{tabular}{@{}l}
reverse'' :: {\char 91}a{\char 93} -> {\char 91}a{\char 93}
\end{tabular}]
{\haddockbegindoc
\section*{Description}
Reverse a given list in o(n) time\par
\subsection*{Example}
\begin{quote}
{\haddockverb\begin{verbatim}
>>> reverse'' [1,2,3,4]
[4,3,2,1]

\end{verbatim}}
\end{quote}}
\end{haddockdesc}