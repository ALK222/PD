\haddockmoduleheading{Traductor}
\label{module:Traductor}
\haddockbeginheader
{\haddockverb\begin{verbatim}
module Traductor (
    NodeElement, Traduction, Bucket, Hash, leaf, genEmptyTree, insert,
    insertList, inorder, busqueda, busquedaLista, evens, odds, hash,
    calcMeanWordCount, calcMeanWordCountExample, rellenadoPorFichero,
    rellenadoPorFicheroFijo, lecturaPalabras, mostrarTraducciones, demoMode
  ) where\end{verbatim}}
\haddockendheader

Práctica final de Programación Declarativa (PD) de Ingeniería Infromática de la UCM\par
\section{Types}
\begin{haddockdesc}
\item[\begin{tabular}{@{}l}
data NodeElement
\end{tabular}]
{\haddockbegindoc
Nodo de la tabla de traduccion usando un hash con colisiones con valores {\char 91}0 .. 9{\char 93} para guardar valores\par}
\end{haddockdesc}
\begin{haddockdesc}
\item[\begin{tabular}{@{}l}
instance Show NodeElement
\end{tabular}]
\end{haddockdesc}
\begin{haddockdesc}
\item[\begin{tabular}{@{}l}
type Traduction = (String, String)
\end{tabular}]
{\haddockbegindoc
Alias para las tuplas usadas para representar traducciones\par}
\end{haddockdesc}
\begin{haddockdesc}
\item[\begin{tabular}{@{}l}
type Bucket = {\char 91}Traduction{\char 93}
\end{tabular}]
{\haddockbegindoc
Elementos de la tabla de traducción asignados a un valor concreto, se guardarán varias traducciones distintas\par}
\end{haddockdesc}
\begin{haddockdesc}
\item[\begin{tabular}{@{}l}
data Hash a
\end{tabular}]
{\haddockbegindoc
Árbol para almacenar todos los nodos, puede ser vacío o un nodo con ramas a ambos lados. Estas ramas también pueden ser vacias\par}
\end{haddockdesc}
\begin{haddockdesc}
\item[\begin{tabular}{@{}l}
instance Show a => Show Hash a
\end{tabular}]
\end{haddockdesc}
\section{Functions}
\subsection{Generators}
\begin{haddockdesc}
\item[\begin{tabular}{@{}l}
leaf :: a -> Hash a
\end{tabular}]
{\haddockbegindoc
\textbf{Description}\par
Genera un árbol que solo tiene raiz, una hoja\par
\subsubsection*{\textbf{Examples}}
\begin{quote}
{\haddockverb\begin{verbatim}
>>> leaf (NE 1 [])
[1] -> []

\end{verbatim}}
\end{quote}}
\end{haddockdesc}
\begin{haddockdesc}
\item[\begin{tabular}{@{}l}
genEmptyTree :: {\char 91}Int{\char 93} -> Hash NodeElement
\end{tabular}]
{\haddockbegindoc
\textbf{Description}\par
Genera un árbol binario de n elementos en una lista para meter las palabras dadas\par}
\end{haddockdesc}
\subsection{Tree and List manipulators}
\begin{haddockdesc}
\item[\begin{tabular}{@{}l}
insert :: Hash NodeElement -> Int -> Traduction -> Hash NodeElement
\end{tabular}]
{\haddockbegindoc
\textbf{Description}\par
Inserta un elemento en el indice dado\par
\subsubsection*{\textbf{Examples}}
\begin{quote}
{\haddockverb\begin{verbatim}
>>> insert (Node (leaf (NE 0 [])) (NE 1 []) Nil) 1 ("a","b")
[0] -> []
[1] -> [("a","b")]

\end{verbatim}}
\end{quote}
\begin{quote}
{\haddockverb\begin{verbatim}
>>> insert Nil 1 ("a","b")
[1] -> [("a","b")]

\end{verbatim}}
\end{quote}}
\end{haddockdesc}
\begin{haddockdesc}
\item[\begin{tabular}{@{}l}
insertList :: {\char 91}Traduction{\char 93} -> Traduction -> {\char 91}Traduction{\char 93}
\end{tabular}]
{\haddockbegindoc
\textbf{Description}\par
Inserción ordenada de traducciones en una lista.\par}
\end{haddockdesc}
\begin{haddockdesc}
\item[\begin{tabular}{@{}l}
inorder :: Show a => Hash a -> String
\end{tabular}]
{\haddockbegindoc
\textbf{Description}\par
Muestra el inorden del árbol dado\par}
\end{haddockdesc}
\begin{haddockdesc}
\item[\begin{tabular}{@{}l}
busqueda :: Hash NodeElement -> Int -> String -> String
\end{tabular}]
{\haddockbegindoc
\textbf{Description}\par
Busca una palabra en el listado dentro del árbol dado.\par
Usa el índice para encontrar el nodo del listado\par
\subsubsection*{\textbf{Examples}}
\begin{quote}
{\haddockverb\begin{verbatim}
>>> busqueda (insert genEmptyTree (hash "hola") ("hola", "hello")) (hash "hola") "hola"
"hello"

\end{verbatim}}
\end{quote}
\begin{quote}
{\haddockverb\begin{verbatim}
>>> busqueda (genEmptyTree) (hash "hola") "hola"
"*** Exception: Palabra no encontrada
CallStack (from HasCallStack):
  error, called at hash.hs:190:28 in main:Traductor

\end{verbatim}}
\end{quote}}
\end{haddockdesc}
\begin{haddockdesc}
\item[\begin{tabular}{@{}l}
busquedaLista :: {\char 91}Traduction{\char 93} -> String -> String
\end{tabular}]
{\haddockbegindoc
\textbf{Description}\par
Busqueda de una palabra en un listado de traducciones para conseguir la traducción.\par
\subsubsection*{\textbf{Examples}}
\begin{quote}
{\haddockverb\begin{verbatim}
>>> busquedaLista [] "hola"
"Palabra no encontrada"

\end{verbatim}}
\end{quote}
\begin{quote}
{\haddockverb\begin{verbatim}
>>> busquedaLista [("Hola", "Hello"), ("A", "B")] "Hola"
"Hello"

\end{verbatim}}
\end{quote}}
\end{haddockdesc}
\begin{haddockdesc}
\item[\begin{tabular}{@{}l}
evens :: {\char 91}a{\char 93} -> {\char 91}a{\char 93}
\end{tabular}]
{\haddockbegindoc
\textbf{Description}\par
Función auxiliar de \haddockid{odds} para eliminar los indices pares de una lista\par}
\end{haddockdesc}
\begin{haddockdesc}
\item[\begin{tabular}{@{}l}
odds :: {\char 91}a{\char 93} -> {\char 91}a{\char 93}
\end{tabular}]
{\haddockbegindoc
\textbf{Description}\par
Elimina los indices pares de una lista\par}
\end{haddockdesc}
\subsection{Utilities}
\begin{haddockdesc}
\item[\begin{tabular}{@{}l}
hash :: String -> Int
\end{tabular}]
{\haddockbegindoc
\textbf{Description}\par
Hace un hash de una palabra y se queda con el último digito para usarlo de índice\par
\subsubsection*{\textbf{Examples}}
\begin{quote}
{\haddockverb\begin{verbatim}
>>> hash "Hola"
8

\end{verbatim}}
\end{quote}
\begin{quote}
{\haddockverb\begin{verbatim}
>>> hash "Adios"
0

\end{verbatim}}
\end{quote}}
\end{haddockdesc}
\begin{haddockdesc}
\item[\begin{tabular}{@{}l}
calcMeanWordCount :: String -> Float
\end{tabular}]
{\haddockbegindoc
\textbf{Description}\par
Caclula la media de letras por palabra en una frase.\par
Usa foldl para recorrer cada una de las palabras y añadir sus letras al numero total\par
Mas tarde se divide entre el numero de palabras para sacar la media de letras sin contar los espacios en blanco\par
\subsubsection*{\textbf{Examples}}
\begin{quote}
{\haddockverb\begin{verbatim}
>>> calcMeanWordCount "a b c d"
1

\end{verbatim}}
\end{quote}
\begin{quote}
{\haddockverb\begin{verbatim}
>>> calcMeanWordCount "ABCD"
4

\end{verbatim}}
\end{quote}}
\end{haddockdesc}
\begin{haddockdesc}
\item[\begin{tabular}{@{}l}
calcMeanWordCountExample :: String -> IO ()
\end{tabular}]
{\haddockbegindoc
\textbf{Description}\par
Coje el texto del fichero dado a la función y calcula la media de letras por palabra.\par
\subsubsection*{\textbf{Examples}}
\begin{quote}
{\haddockverb\begin{verbatim}
>>> calcMeanWordCountExample "./pruebas/quijote.txt"
4.5021253

\end{verbatim}}
\end{quote}
\begin{quote}
{\haddockverb\begin{verbatim}
>>> calcMeanWordCountExample "./pruebas/call_of_cthulhu.txt"
4.8737144

\end{verbatim}}
\end{quote}}
\end{haddockdesc}
\begin{haddockdesc}
\item[\begin{tabular}{@{}l}
rellenadoPorFichero :: IO (Hash NodeElement)
\end{tabular}]
{\haddockbegindoc
\textbf{Description}\par
Pide un fichero para generar el árbol de traducciones\par}
\end{haddockdesc}
\begin{haddockdesc}
\item[\begin{tabular}{@{}l}
rellenadoPorFicheroFijo :: IO (Hash NodeElement)
\end{tabular}]
{\haddockbegindoc
\textbf{Description}\par
Lectura del fichero "datos.txt" para cargar un arbol de traducciones\par}
\end{haddockdesc}
\subsection{IO}
\begin{haddockdesc}
\item[\begin{tabular}{@{}l}
lecturaPalabras :: String -> IO String
\end{tabular}]
{\haddockbegindoc
\textbf{Description}\par
Lee palabras por el teclado y devuelve un String con todas las palabras introducidas\par}
\end{haddockdesc}
\begin{haddockdesc}
\item[\begin{tabular}{@{}l}
mostrarTraducciones :: IO ()
\end{tabular}]
{\haddockbegindoc
\textbf{Description}\par
Traduce las palabras dadas por el usuario por teclado.\par
Genera un árbol vacío, lo llena con un archivo de traducciones y lo usa para traducir una serie de palabras dadas\par}
\end{haddockdesc}
\begin{haddockdesc}
\item[\begin{tabular}{@{}l}
demoMode :: IO ()
\end{tabular}]
{\haddockbegindoc
\textbf{Description}\par
Modo para probar todas las funciones implementadas en el módulo\par}
\end{haddockdesc}