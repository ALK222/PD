\haddockmoduleheading{Labo2}
\label{module:Labo2}
\haddockbeginheader
{\haddockverb\begin{verbatim}
module Labo2 (
    cuadrados, cuadradoInverso, rsumcos, sumMenores,
    siguientePrimo, factores, isPrime, iguales, menor, mayorA, ex,
    filter2, filters, mapx
  ) where\end{verbatim}}
\haddockendheader

\section{EJERCICIO 1}
\subsection{A)}
\begin{haddockdesc}
\item[\begin{tabular}{@{}l}
cuadrados :: (Num b, Enum b) => b -> {\char 91}b{\char 93}
\end{tabular}]
{\haddockbegindoc
\section*{Descripción}
Coje un números y devuelve una lista con los cuadrados desde 0 hasta el número\par
\subsection*{Ejemplos}
\begin{quote}
{\haddockverb\begin{verbatim}
>>> cuadrados 3
[0,1,4,9]

\end{verbatim}}
\end{quote}}
\end{haddockdesc}
\subsection{B)}
\begin{haddockdesc}
\item[\begin{tabular}{@{}l}
cuadradoInverso :: (Num a, Enum a) => a -> {\char 91}(a, a){\char 93}
\end{tabular}]
{\haddockbegindoc
\section*{Descripción}
Coje un números y devuelve una lista con los cuadrados desde 0 hasta el número en orden inverso y emparejado con su número inicial\par
\subsection*{Ejemplos}
\begin{quote}
{\haddockverb\begin{verbatim}
>>> cuadradoInverso 3
[(3,9),(2,4),(1,1),(0,0)]

\end{verbatim}}
\end{quote}}
\end{haddockdesc}
\subsection{C)}
\begin{haddockdesc}
\item[\begin{tabular}{@{}l}
rsumcos :: (Floating a, Enum a) => a -> a
\end{tabular}]
{\haddockbegindoc
\section*{Descripción}
Sumatorio del coseno de 1 hasta el número dado\par
\subsection*{Ejemplos}
\begin{quote}
{\haddockverb\begin{verbatim}
>>> rsumcos 90
2592.852064773843

\end{verbatim}}
\end{quote}}
\end{haddockdesc}
\subsection{D)}
\begin{haddockdesc}
\item[\begin{tabular}{@{}l}
sumMenores :: Integral a => a -> a
\end{tabular}]
{\haddockbegindoc
\section*{Descripción}
Sumatorio de los números menores que el dado que sean multiplos de 5 o 3\par
\subsection*{Ejemplos}
\begin{quote}
{\haddockverb\begin{verbatim}
>>> sumMenores 10
33

\end{verbatim}}
\end{quote}}
\end{haddockdesc}
\subsection{E)}
\begin{haddockdesc}
\item[\begin{tabular}{@{}l}
siguientePrimo :: Integral a => a -> a
\end{tabular}]
{\haddockbegindoc
\section*{Descripción}
calcula el siguiente número primo a uno dado.\par
Usa \haddockid{isPrime}\par
\subsection*{Ejemplos}
\begin{quote}
{\haddockverb\begin{verbatim}
>>> siguientePrimo 10
11

\end{verbatim}}
\end{quote}}
\end{haddockdesc}
\begin{haddockdesc}
\item[\begin{tabular}{@{}l}
factores :: Integral a => a -> {\char 91}a{\char 93}
\end{tabular}]
{\haddockbegindoc
\section*{Descripción}
Calcula todos los divisores de un número dado\par
\subsection*{Ejemplos}
\begin{quote}
{\haddockverb\begin{verbatim}
>>> factores 10
[1,2,5,10]

\end{verbatim}}
\end{quote}}
\end{haddockdesc}
\begin{haddockdesc}
\item[\begin{tabular}{@{}l}
isPrime :: Integral a => a -> Bool
\end{tabular}]
{\haddockbegindoc
\section*{Descripción}
Comprueba si un número es primo\par
Usa \haddockid{factores}\par
\subsection*{Ejemplos}
\begin{quote}
{\haddockverb\begin{verbatim}
>>> isPrime 10
False

\end{verbatim}}
\end{quote}
\begin{quote}
{\haddockverb\begin{verbatim}
>>> isPrime 5
True

\end{verbatim}}
\end{quote}}
\end{haddockdesc}
\section{EJERCICIO 2}
\subsection{A)}
\begin{haddockdesc}
\item[\begin{tabular}{@{}l}
iguales :: Eq b => (a -> b) -> (a -> b) -> a -> a -> Bool
\end{tabular}]
{\haddockbegindoc
\section*{Descripción}
Comprueba si dos funciones son iguales para un ranfo dado de valores\par
\subsection*{Ejemplos}
\begin{quote}
{\haddockverb\begin{verbatim}
>>> iguales siguientePrimo sumMenores  0 10
False

\end{verbatim}}
\end{quote}
\begin{quote}
{\haddockverb\begin{verbatim}
>>> iguales sumMenores sumMenores  0 10
True

\end{verbatim}}
\end{quote}}
\end{haddockdesc}
\subsection{B)}
\begin{haddockdesc}
\item[\begin{tabular}{@{}l}
menor :: (Enum a, Num a) => a -> (a -> Bool) -> a
\end{tabular}]
{\haddockbegindoc
\section*{Descripción}
Devuelve el menor numero mayor que n que compla una función dada\par
\subsection*{Ejemplos}
\begin{quote}
{\haddockverb\begin{verbatim}
>>> menor 10 isPrime
11

\end{verbatim}}
\end{quote}}
\end{haddockdesc}
\subsection{C)}
\begin{haddockdesc}
\item[\begin{tabular}{@{}l}
mayorA :: Enum a => a -> a -> (a -> Bool) -> a
\end{tabular}]
{\haddockbegindoc
\section*{Descripción}
Mayor número en un intervalo dado que cumple una función dada\par
\subsection*{Ejemplos}
\begin{quote}
{\haddockverb\begin{verbatim}
>>> mayorA 10 20 isPrime
19

\end{verbatim}}
\end{quote}}
\end{haddockdesc}
\subsection{D)}
\begin{haddockdesc}
\item[\begin{tabular}{@{}l}
ex :: Enum a => a -> a -> (a -> Bool) -> Bool
\end{tabular}]
{\haddockbegindoc
\section*{Descripción}
Comprueba si existe algún número en el intervalo dado que verifique una función\par
\subsection*{Ejemplos}
\begin{quote}
{\haddockverb\begin{verbatim}
>>> ex 14 16 isPrime
False

\end{verbatim}}
\end{quote}
\begin{quote}
{\haddockverb\begin{verbatim}
>>> ex 10 15 isPrime
True

\end{verbatim}}
\end{quote}}
\end{haddockdesc}
\section{EJERCICIO 3}
\subsection{A)}
\begin{haddockdesc}
\item[\begin{tabular}{@{}l}
filter2 :: {\char 91}a{\char 93} -> (a -> Bool) -> (a -> Bool) -> ({\char 91}a{\char 93}, {\char 91}a{\char 93})
\end{tabular}]
{\haddockbegindoc
\section*{Descripción}
Devuelve dos listas con los elementos de una lista dada que cumplen dos funciones\par
\subsection*{Ejemplos}
\begin{quote}
{\haddockverb\begin{verbatim}
>>> filter2 [1..10] odd even
([1,3,5,7,9],[2,4,6,8,10])

\end{verbatim}}
\end{quote}}
\end{haddockdesc}
\subsection{B)}
\begin{haddockdesc}
\item[\begin{tabular}{@{}l}
filters :: {\char 91}a{\char 93} -> {\char 91}a -> Bool{\char 93} -> {\char 91}{\char 91}a{\char 93}{\char 93}
\end{tabular}]
{\haddockbegindoc
\section*{Descripción}
Devuelve una lista con las listas de numeros que cumplen unas funciones dadas en formato de lista\par
\subsection*{Ejemplos}
\begin{quote}
{\haddockverb\begin{verbatim}
>>> filters [1..10] [odd, even, isPrime]
[[1,3,5,7,9],[2,4,6,8,10],[2,3,5,7]]

\end{verbatim}}
\end{quote}}
\end{haddockdesc}
\subsection{C)}
\begin{haddockdesc}
\item[\begin{tabular}{@{}l}
mapx :: t -> {\char 91}t -> b{\char 93} -> {\char 91}b{\char 93}
\end{tabular}]
{\haddockbegindoc
\section*{Descripción}
Devuelve una lista de valores resultado de aplicar a un valor un listado de funciones. Todos los tipos de las funciones de la lista deben ser iguales\par
\subsection*{Ejemplos}
\begin{quote}
{\haddockverb\begin{verbatim}
>>> mapx 10 [even, odd]
[True,False]

\end{verbatim}}
\end{quote}
\begin{quote}
{\haddockverb\begin{verbatim}
>>> mapx 10 [siguientePrimo, sumMenores]
[11,33]

\end{verbatim}}
\end{quote}}
\end{haddockdesc}